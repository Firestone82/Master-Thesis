\documentclass[czech,master]{diploma}

% Standard packages
\usepackage[autostyle=true,czech=quotes]{csquotes} % correct quotation typesetting, support for biblatex
\usepackage[backend=biber, style=iso-numeric, alldates=iso]{biblatex} % bibliography
\usepackage{dcolumn} % table columns with numeric values
\usepackage{subfig} % macros for "subfigures" and "subtables"

% Custom packages
\usepackage{float} % better figure placement (H)
\usepackage{bookmark} % Required for generating PDF bookmarks
\usepackage{hyperref} % Links in PDF
\usepackage{subcaption} % For multiple images in one float
\usepackage{tikz} % For drawing in LaTeX
\usepackage{amsmath} % For improved equation typesetting
\usepackage{tablefootnote} % For footnotes in tables
\usepackage{color} % For colored text (e.g., TODO notes)
\usepackage{array} % For advanced table formatting
\usepackage[cpp]{diplomalst}
\usepackage{fontspec}

% Table of contents depth - TODO: Remove in final version!
\setcounter{tocdepth}{3}

\renewcommand{\lstlistingname}{Zdrojový Kód} % Convert "Listing" to "Zdrojový Kód"
\newcolumntype{d}[1]{D{,}{,}{#1}} % New type of table column with numbers aligned by decimal comma

% Required thesis metadata
\ThesisAuthor{Bc. Pavel Mikula}
\ThesisSupervisor{Mgr. Ing. Michal Krumnikl, Ph.D.}
\CzechThesisTitle{Integrace velkých jazykových modelů do systému Kelvin}
\EnglishThesisTitle{Integrating large language models into the Kelvin system}
\SubmissionYear{2025}

\ThesisAssignmentFileName{resources/ThesisSpecification_MIK0486.pdf}

\CzechAbstract{\TODO{Přidat český abstrakt práce}}
\CzechKeywords{\TODO{Přidat řeské keywordy práce}}

\EnglishAbstract{\TODO{Přidat anglický abstrakt práce}}
\EnglishKeywords{\TODO{Přidat anglické keywordy práce}}

% TODO: add acronyms used in the thesis
\AddAcronym{LLM}{Large Language Model}
\AddAcronym{PR}{Pull Request}
\AddAcronym{API}{Application Programming Interface}
\AddAcronym{UI}{User Interface}
\AddAcronym{UX}{User Experience}
\AddAcronym{CI}{Continuous Integration}
\AddAcronym{CD}{Continuous Deployment}
\AddAcronym{RNN}{Recurrent Neural Network}
\AddAcronym{GPT}{Generative Pre-trained Transformer}

\Acknowledgement{Rád bych na tomto místě poděkoval všem, kteří mi s prací pomohli, protože bez nich by tato práce nevznikla. \TODO: Doplnit konkrétní jména a poděkování.}

\addbibresource{resources/sauce.bib}

% Useful to.do command for marking places in the text that need to be completed
\newcommand{\TODO}[1]{\textbf{\textcolor{red}{TODO:}} #1}

% Custom command for inline code snippets
\definecolor{codegray}{gray}{0.95}
\newcommand{\code}[1]{%
    \fcolorbox{black}{codegray}{\texttt{#1}}%
}

\newcommand{\tableref}[1]{(\hyperref[#1]{Tabulka~\ref*{#1}})}

%
% ==] Main document [==
%

\begin{document}
    \MakeTitlePages

    % Pictures
    \listoffigures
    \clearpage

    % Tables
    \listoftables
    \clearpage

    % Source codes
    \renewcommand{\lstlistlistingname}{Seznam zdrojových kódů}
    \addcontentsline{toc}{chapter}{Seznam zdrojových kódů}
    \lstlistoflistings

    % Introduction
    \clearpage
    \chapter{Úvod}
\label{ch:introduction}

Výuka programování je v současnosti běžně podporována školními informačními systémy, které slouží k odevzdávání úloh, testování studentů a správě jejich hodnocení.
Tyto systémy výrazně usnadňují organizaci výuky, zejména při větším počtu studentů.
Jednou z nejnáročnějších částí však zůstává samotné hodnocení zdrojových kódů, při kterém je nutné posuzovat nejen správnou funkčnost řešení, ale také jeho strukturu, metodiku a případné chyby.

Tato diplomová práce se zaměřuje na rozšíření školního systému Kelvin o možnost automatické analýzy odevzdaných zdrojových kódů pomocí velkých jazykových modelů (LLM).
Cílem tohoto rozšíření není nahradit učitele při hodnocení, ale poskytnout mu podpůrný nástroj, který mu pomůže rychleji se zorientovat v řešení a upozorní na problematické části kódu.

\section{Motivace práce}
\label{sec:motivation}

Systém Kelvin vznikl původně jako studentský projekt pro správu odevzdávání úloh a testování studentů.
Postupně se ale vyvinul v plnohodnotný systém používaný při výuce, kde jeho vývoj pokračuje i nadále.
Neboť se jedná o open-source projekt, který se stále rozšiřuje, zdokonaluje a přizpůsobuje potřebám výuky.

Nicméně kvůli rostoucímu počtu studentů a komplexnosti úloh se stává stále náročnější pro učitele efektivně hodnotit odevzdané kódy.
Učitelé musí projít velké množství řešení, přičemž jednotlivé kódy mohou být velmi různorodé, a to nejen z hlediska funkčnosti, ale i z hlediska stylu psaní a struktury.
To může vést k tomu, že učitelé nemají dostatek času na důkladné hodnocení každého řešení, což může negativně ovlivnit kvalitu zpětné vazby pro studenty.

Motivací této práce je proto snaha tento proces podpořit automatizovaným nástrojem, který by dokázal analyzovat odevzdaný zdrojový kód a vyučujícímu nabídnout strukturované připomínky.
Dalším důvodem je ale samotný vývoj systému Kelvin, který je dlouhodobě rozšiřován studenty nebo vyučujícími, a nese s sebou určitý technický dluh.

\section{Cíle práce}
\label{sec:goals}

Hlavním cílem diplomové práce je návrh a implementace modulu založeného na velkých jazykových modelech, který bude integrován přímo do systému Kelvin.
Tento modul bude po odevzdání zdrojového kódu studentem asynchronně zpracovávat jeho řešení a generovat automatickou analýzu.

Výstupem této analýzy budou zejména:
\begin{itemize}
    \item komentáře vztahující se ke konkrétním částem nebo řádkům zdrojového kódu
    \item stručné shrnutí celkového řešení
\end{itemize}

Důraz je kladen na to, aby výsledky analýzy sloužily jako podpůrný nástroj pro učitele.
Učitel má možnost na vygenerované komentáře reagovat, upravovat je nebo je použít jako podklad pro výsledné hodnocení.
Hodnocení a zpětná vazba ze strany učitele mohou být dále využity pro budoucí zlepšování celého systému.

Součástí cíle práce je také porovnání různých přístupů k nasazení jazykových modelů, volba vhodného modelu a návrh práce s prompty tak, aby systém poskytoval co nejkvalitnější výstupy při rozumných nárocích na výpočetní prostředky.

\section{Struktura práce}
\label{sec:structure}

\TODO{Dávat zde nějakou strukturu práce? Nebo je to jasné z obsahu.}

\endinput

    % Main chapters
    \chapter{System Kelvin}
\label{ch:kelvin}

Tato kapitola se věnuje školnímu systému Kelvin, který slouží jako základní platforma pro odevzdávání a hodnocení programátorských úloh.
Cílem kapitoly je přiblížit účel systému, jeho architekturu, způsob nasazení a především vývojová omezení, která významně ovlivnila návrh a implementaci řešení popsaného v této diplomové práci.

\section{Účel systému}
\label{sec:kelvin-purpose}

Hlavním účelem systému Kelvin je podpora výuky programování, především v jazycích C, C++, Python nebo Rust.
Systém poskytuje jednotné webové rozhraní, prostřednictvím kterého mohou studenti odevzdávat svá řešení úloh.
Vyučujícím následně umožňuje tato řešení kontrolovat, poskytovat k nim zpětnou vazbu a udělovat bodové hodnocení.

Typický, zjednodušený scénář použití systému probíhá následovně:
\begin{itemize}
    \item vyučující vytvoří úlohu a zveřejní ji studentům,
    \item student odevzdá své řešení úlohy ve formě zdrojového kódu,
    \item systém řešení uloží a případně spustí automatické testy, které vygenerují základní zpětnou vazbu,
    \item vyučující provede kontrolu zdrojového kódu, přidá komentáře k jednotlivým částem řešení a udělí bodové hodnocení,
    \item student obdrží zpětnou vazbu a může se případně zapojit do další diskuse s vyučujícím ohledně svého řešení.
\end{itemize}

Významnou výhodou systému Kelvin je možnost přidávat komentáře přímo ke konkrétním řádkům zdrojového kódu.
Díky tomu může vyučující upozornit na konkrétní chyby, nevhodné konstrukce nebo navrhnout možná zlepšení přímo v kontextu řešení, což je pro studenty velmi přínosné a srozumitelné.

Systém Kelvin je využíván nejen pro klasické domácí úlohy, ale také pro testy, kvízy a další formy hodnocení, kde není nutné pracovat se zdrojovým kódem.

\section{Architektura systému}
\label{sec:kelvin-architecture}

Z architektonického hlediska je systém Kelvin realizován jako webová aplikace, která je rozdělena na backendovou a frontendovou část.
Backend je implementován v programovacím jazyce Python s využitím frameworku Django, který zajišťuje zpracování požadavků, práci s databází a autentizaci uživatelů.

Frontendová část systému je z historického hlediska výrazně komplikovanější.
Původní implementace byla postavena na Django templatech, které byly v průběhu času postupně rozšiřovány o komponenty psané ve frameworku Svelte.
V současné době však probíhá migrace ze Svelte na framework Vue, což má za následek, že se ve zdrojovém kódu systému současně nachází mix třech různých frontendových technologií, konkrétně tedy Django templaty, Svelte komponenty a Vue komponenty.

Současná kombinace více frontendových technologií výrazně zvyšuje složitost vývoje i údržby celého systému.
Z tohoto důvodu je přidávání nových funkcionalit na frontendové straně omezeno interními pravidly, jejichž cílem je zabránit dalšímu rozšiřování původní Svelte implementace a postupně směřovat celý systém k jednotnému řešení postavenému na frameworku Vue.

Komunikace mezi frontendem a backendem probíhá prostřednictvím aplikačního rozhraní API, které je realizováno pomocí frameworku Django Ninja.
Toto rozhraní umožňuje asynchronní zpracování požadavků a zároveň odděluje prezentační logiku od aplikační logiky systému.

\begin{figure}[ht]
    \centering
    \includegraphics[width=0.8\textwidth]{resources/diagrams/kelvin-architecture}
    \caption{Zjednodušená architektura systému Kelvin}\label{fig:kelvin-architecture}
    \TODO{Přidat referenci na docs https://mrlvsb.github.io/kelvin/intro/architecture}
\end{figure}

\section{Nasazení a provoz systému}
\label{sec:kelvin-deployment}

Pro produkční nasazení je systém Kelvin distribuován pomocí kontejnerizační technologie Docker, přičemž jednotlivé části aplikace jsou spravovány nástrojem Docker Compose.
Tento přístup umožňuje zajistit jednotné běhové prostředí pro vývoj i produkci a zároveň usnadňuje správu závislostí aplikace.

Proces nasazení systému je automatizován pomocí nástroje GitHub Actions, který po každém úspěšném sloučení změn do hlavní větve repozitáře provede sestavení Docker obrazů a jejich nasazení na produkční server.
Automatizovaný deployment snižuje riziko chyb při manuálním nasazování a zajišťuje, že produkční prostředí odpovídá aktuálnímu stavu zdrojového kódu.

Využití Dockeru a Docker Compose zároveň zjednodušuje provozní správu systému a umožňuje jeho další rozšiřování bez nutnosti zásadních změn v infrastruktuře.

\section{Vývojová omezení a proces}
\label{sec:kelvin-development}

Vývoj systému Kelvin je ovlivněn několika zásadními omezeními, která vyplývají jak z jeho historického vývoje, tak z organizačních podmínek jeho správy a údržby.
Tato omezení mají přímý dopad na návrh i samotnou implementaci řešení popsaného v této diplomové práci.

Jedním z hlavních pravidel vývoje je požadavek na malé a atomické pull requesty.
Každý pull request by měl obsahovat maximálně přibližně 500 změněných řádků kódu a měl by se zaměřovat pouze na jednu konkrétní změnu nebo problém.
Cílem tohoto přístupu je zjednodušení procesu code review a snížení rizika zavlečení chyb do produkčního systému.
V praxi to však znamená nutnost rozdělovat i logicky související úpravy do více menších kroků, což může celý vývojový proces prodlužovat.

Dalším významným omezením je probíhající migrace frontendové části systému ze frameworku Svelte na framework Vue.
Při úpravách nebo přidávání nových funkcionalit by neměly vznikat nové Svelte komponenty.
V případě rozsáhlejších změn je tak často nutné nejprve přepsat existující část aplikace do Vue a teprve následně provést požadovanou úpravu.
Tento postup výrazně zvyšuje časovou i technickou náročnost vývoje, zejména u funkcionalit zasahujících do uživatelského rozhraní.

Důvodem této migrace je zejména snaha o sjednocení technologického stacku, zlepšení dlouhodobé udržovatelnosti a snížení vstupní bariéry pro nové vývojáře systému.

Proces schvalování změn představuje další omezení vývoje.
Code review je v současné době zajišťováno omezeným počtem vyučujících, kteří se systému věnují vedle svých dalších povinností.
Z tohoto důvodu se může stát, že zpracování pull requestu trvá i několik týdnů.
Tento fakt má přímý dopad na tempo vývoje a vyžaduje pečlivé plánování jednotlivých kroků implementace s dostatečným časovým předstihem.

\subsection{Historický vývoj systému}
\label{subsec:kelvin-history}

Systém Kelvin vznikl původně jako obyčejný školní projekt od studenta Dana Trnky \footnote[1]{https://github.com/trnila}, ale v průběhu let byl postupně rozšiřován dalšími studenty a vyučujícími.
Tento způsob vývoje bez jednotné dlouhodobé architektonické vize vedl ke vzniku technického dluhu.
Ten se v systému projevuje například ve formě dlouhých a obtížně čitelných souborů, nedostatečného nebo chybějícího typování a míchání různých programátorských přístupů.

Uvedené skutečnosti mají výrazný vliv na implementaci nových funkcionalit.
Před přidáním nové aplikační logiky je často nutné provést částečný refaktoring existujícího kódu, aby byla zachována alespoň základní úroveň čitelnosti a udržovatelnosti systému.
Tato nutnost dále zpomaluje vývoj, ale zároveň je nezbytná pro dlouhodobou stabilitu a rozšiřitelnost systému Kelvin.

\endinput
    \chapter{Velké jazykové model}
\label{ch:llm}

\endinput
    \chapter{Výběr modelů a vhodnost pro integraci}
\label{ch:selection}

\endinput
    \chapter{Impementace do systému Kelvin}
\label{ch:implementation}

\endinput
    \chapter{Vyhodnocení řešení}
\label{ch:summary}

\endinput
    \chapter{Doporučení pro další rozvoj a implementaci}
\label{ch:suggestion}

\endinput
    \chapter{Závěr}
\label{ch:conclusion}

\endinput

    \appendix
    \input{chapters/9-Appendix}

    % Bibliography
    \printbibliography[title={Literatura}, heading=bibintoc]

\end{document}
