\chapter{Úvod}
\label{ch:introduction}

Výuka programování je v současnosti běžně podporována školními informačními systémy, které slouží k odevzdávání úloh, testování studentů a správě jejich hodnocení.
Tyto systémy výrazně usnadňují organizaci výuky, zejména při větším počtu studentů.
Jednou z nejnáročnějších částí však zůstává samotné hodnocení zdrojových kódů, při kterém je nutné posuzovat nejen správnou funkčnost řešení, ale také jeho strukturu, metodiku a případné chyby.

Tato diplomová práce se zaměřuje na rozšíření školního systému Kelvin o možnost automatické analýzy odevzdaných zdrojových kódů pomocí velkých jazykových modelů (LLM).
Cílem tohoto rozšíření není nahradit učitele při hodnocení, ale poskytnout mu podpůrný nástroj, který mu pomůže rychleji se zorientovat v řešení a upozorní na problematické části kódu.

\section{Motivace práce}
\label{sec:motivation}

Systém Kelvin vznikl původně jako studentský projekt pro správu odevzdávání úloh a testování studentů.
Postupně se ale vyvinul v plnohodnotný systém používaný při výuce, kde jeho vývoj pokračuje i nadále.
Neboť se jedná o open-source projekt, který se stále rozšiřuje, zdokonaluje a přizpůsobuje potřebám výuky.

Nicméně kvůli rostoucímu počtu studentů a komplexnosti úloh se stává stále náročnější pro učitele efektivně hodnotit odevzdané kódy.
Učitelé musí projít velké množství řešení, přičemž jednotlivé kódy mohou být velmi různorodé, a to nejen z hlediska funkčnosti, ale i z hlediska stylu psaní a struktury.
To může vést k tomu, že učitelé nemají dostatek času na důkladné hodnocení každého řešení, což může negativně ovlivnit kvalitu zpětné vazby pro studenty.

Motivací této práce je proto snaha tento proces podpořit automatizovaným nástrojem, který by dokázal analyzovat odevzdaný zdrojový kód a vyučujícímu nabídnout strukturované připomínky.
Dalším důvodem je ale samotný vývoj systému Kelvin, který je dlouhodobě rozšiřován studenty nebo vyučujícími, a nese s sebou určitý technický dluh.

\section{Cíle práce}
\label{sec:goals}

Hlavním cílem diplomové práce je návrh a implementace modulu založeného na velkých jazykových modelech, který bude integrován přímo do systému Kelvin.
Tento modul bude po odevzdání zdrojového kódu studentem asynchronně zpracovávat jeho řešení a generovat automatickou analýzu.

Výstupem této analýzy budou zejména:
\begin{itemize}
    \item komentáře vztahující se ke konkrétním částem nebo řádkům zdrojového kódu
    \item stručné shrnutí celkového řešení
\end{itemize}

Důraz je kladen na to, aby výsledky analýzy sloužily jako podpůrný nástroj pro učitele.
Učitel má možnost na vygenerované komentáře reagovat, upravovat je nebo je použít jako podklad pro výsledné hodnocení.
Hodnocení a zpětná vazba ze strany učitele mohou být dále využity pro budoucí zlepšování celého systému.

Součástí cíle práce je také porovnání různých přístupů k nasazení jazykových modelů, volba vhodného modelu a návrh práce s prompty tak, aby systém poskytoval co nejkvalitnější výstupy při rozumných nárocích na výpočetní prostředky.

\section{Struktura práce}
\label{sec:structure}

\TODO{Dávat zde nějakou strukturu práce? Nebo je to jasné z obsahu.}

\endinput