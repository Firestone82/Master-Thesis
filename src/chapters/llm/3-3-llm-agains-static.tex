\section{Porovnání s tradičními statickými analyzátory}
\label{sec:llm-vs-static}

% https://agileseekers.com/blog/using-static-code-analysis-tools-as-part-of-sprint-reviews
% https://blog.codacy.com/static-code-analysis-tools
% https://okt.inf.szte.hu/tesztalap/gyakorlat/gyak07/

Statická analýza kódu představuje techniku, při které je zdrojový kód analyzován bez jeho spuštění.
Cílem je identifikovat chyby, potenciální problémy, bezpečnostní rizika nebo porušení konvenčních pravidel.
Tyto nástroje jsou běžnou součástí profesionálního vývoje softwaru a často bývají integrovány do CI/CD pipeline, vývojových prostředí nebo procesu code review.

Zásadní vlastností statické analýzy je, že nepracuje s reálným během programu, ale s jeho strukturou.
Analyzátor využívá formální pravidla, kontrolu toku programu (control flow analysis), datové závislosti nebo předem definované heuristiky.
Výsledkem je deterministická analýza, která je při stejném vstupu opakovatelná a konzistentní.

V kontextu této práce je důležité pochopit, že velké jazykové modely fungují na zcela odlišném principu.
LLM neprovádí formální analýzu abstraktního syntaktického stromu ani kontrolu datových toků.
Místo toho generuje text na základě pravděpodobnostních vzorů, které se naučil během trénování.
To vede k odlišným vlastnostem i omezením obou přístupů.

\subsection{Typy statických analyzátorů}
\label{subsec:static-types}

Statické analyzátory (SA) lze rozdělit do několika kategorií podle rozsahu a hloubky analýzy.

\begin{enumerate}
    \item \textbf{Kompilátorová analýza (základní warningy)} –
    Kompilátory jako GCC nebo Clang generují upozornění během překladu programu.
    Typicky se jedná o detekci nevyužitých proměnných, potenciálně nebezpečných konstrukcí nebo porušení jazykových pravidel.
    Tato analýza je úzce svázána s konkrétním jazykem a jeho formální specifikací.

    \item \textbf{Lintery a pokročilé nástroje (např. clang-tidy, cppcheck)} –
    Tyto nástroje rozšiřují základní kontrolu o dodatečná pravidla.
    Upozorňují na nedodržení dobrých programátorských praktik, stylové nesrovnalosti nebo potenciální logické chyby.
    Často umožňují konfigurovat vlastní pravidla.

    \item \textbf{Enterprise-grade analyzátory (např. SonarQube, CodeQL)} –
    Pokročilé nástroje určené pro větší projekty.
    Kromě detekce chyb generují metriky kvality kódu, analyzují závislosti mezi moduly a dokáží identifikovat bezpečnostní zranitelnosti.
    Často poskytují přehledové reporty a sledování trendů v čase.
\end{enumerate}

\begin{table}[H]
    \centering
    \resizebox{\textwidth}{!}{%
        \begin{tabular}{lll}
            \toprule
            \textbf{Vlastnost} & \textbf{Statický analyzátor} & \textbf{Velký jazykový model} \\
            \midrule
            Detekce syntaktických chyb & Ano (deterministicky) & Omezeně \\
            Formální verifikace pravidel & Ano & Ne \\
            Analýza řízení toku (control flow) & Ano & Omezeně / heuristicky \\
            Práce s kontextem zadání & Ne & Ano \\
            Hodnocení čitelnosti a stylu & Omezeně (dle pravidel) & Ano \\
            Generování vysvětlení v přirozeném jazyce & Omezeně (technická hlášení) & Ano \\
            Determinističnost výstupu & Ano & Ne\tablefootnote{LLM generuje výstup pravděpodobnostně. Výsledek může být ovlivněn parametry inference a nemusí být plně reprodukovatelný.\label{tfn:llm}} \\
            Kontrola bezpečnosti & Ano & Ne\footref{tfn:llm} \\
            \bottomrule
        \end{tabular}
    }
    \caption{Porovnání statických analyzátorů a velkých jazykových modelů při analýze zdrojového kódu}
    \label{tab:llm-vs-static}
\end{table}

\subsection{Zásadní rozdíly v přístupu}
\label{subsec:static-vs-llm-differences}

Hlavní rozdíl mezi statickými analyzátory a velkými jazykovými modely spočívá v povaze jejich analýzy.

Statický analyzátor pracuje deterministicky.
Vychází z formální reprezentace programu a aplikace přesně definovaných pravidel.
Při stejném vstupu vždy poskytne stejný výstup.

Oproti tomu LLM generuje odpověď pravděpodobnostně.
Výsledek může být ovlivněn parametry inference (např.\ teplotou) a není zaručena plná reprodukovatelnost.
Model také neprovádí skutečnou kontrolu běhových cest ani formální verifikaci.

Na druhou stranu má LLM schopnost pracovat s kontextem zadání úlohy.
Dokáže interpretovat nejen samotný kód, ale i jeho účel.
Může například upozornit, že řešení sice funguje, ale není optimální vzhledem k požadavkům zadání.
Taková analýza je pro tradiční statický analyzátor velmi obtížná, protože postrádá znalost záměru.

Dalším významným rozdílem je schopnost vysvětlení.
Statické analyzátory typicky poskytují stručné technické hlášení.
LLM je schopen generovat vysvětlení v přirozeném jazyce, které může být pro studenta srozumitelnější.

\subsection{Možnost kombinace přístupů}
\label{subsec:static-llm-combination}

Statické analyzátory a velké jazykové modely by neměly být vnímány jako konkurenční technologie.
Naopak se mohou vzájemně doplňovat.

Statický analyzátor poskytuje formální, deterministickou kontrolu syntaxe a základních pravidel.
LLM může nad těmito výsledky provést vyšší úroveň interpretace, například vysvětlit, proč je konkrétní konstrukce problematická, nebo navrhnout alternativní řešení.

V kontextu školního systému může být kombinace obou přístupů velmi efektivní.
Zatímco automatické testy a statická analýza ověřují formální správnost řešení, LLM může sloužit jako podpůrný nástroj pro generování srozumitelných komentářů a doporučení.

Tento hybridní přístup umožňuje zachovat technickou přesnost a zároveň zvýšit kvalitu zpětné vazby poskytované studentům.

\endinput

